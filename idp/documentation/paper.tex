\documentclass[a4paper]{article}

\usepackage[utf8]{inputenc}
\usepackage[T1]{fontenc}
\usepackage{parskip}
\usepackage{times}
\usepackage{mathtools}
\usepackage{amssymb}
\usepackage{multirow}

\clubpenalty = 10000
\widowpenalty = 10000

\title{Optimal Fitting of Planar Curves to Prescribed Constraints}
\author{Julian Asamer, Julian Brunner}
\date{\today}

\begin{document}

	\maketitle

	\begin{abstract}

		\noindent TODO

	\end{abstract}

	\section{Introduction}

		These days, vector graphics are widely used in both artistic and industrial design. Artists create illustrations, character designs and even fully animated cartoons using vector graphics software. Vector graphics are also used in industrial design, allowing for the smooth, aerodynamic shapes observed in modern cars and airplanes.

		Raster graphics images (for instance, digital photos and most digitally painted artworks) consist of a finite number of pixels (picture elements) which discretely specify the colors at each point of the image. In contrast, vector graphics designs consist of primitives like curves and surfaces, which describe the shapes present in the design using continuous mathematical functions. This has various advantages over the raster approach, since designs do not suffer from discretization artifacts and can, for instance, be arbitrarily scaled without any loss of quality.

		The scope of this project is restricted to two-dimensional vector graphics, but many ideas can be applied to three-dimensional vector graphics as well. All elements in two-dimensional vector graphics are described in terms of curves, so the design process of curves is the single most important part of two-dimensional vector graphics. Unfortunately, designing curves using the tools available through current vector graphics software can be very frustrating. The artist is often unable to make the curve look the way they want it to. To make matters worse it is often unclear why this is the case, as the software seems to put all the necessary tools into the hands of the designer, yet at times, they somehow don't allow the artist to communicate their intention to the software. Thus, the objective of this project is to identify the exact nature and cause of these inadequacies and then specify and implement curve design tools that do not exhibit them.

		% TODO: make sure introduction does not overlap with start of next section regarding both content and wording

	\section{Usability of Curve Design Tools}

		When trying to improve something, it seems advisable to take a step back and thoroughly research the causes of its perceived shortcomings so as to find a way to improve it without making the same or similar mistakes. In the context of curve design tools, this can be achieved by analyzing and modelling the process that people use when designing curves. Once such a model has been obtained, it can be used to propose criteria for good curve design software as well as to analyze potential issues with existing tools.

		\subsection{Curve Design Process}
		\label{section:CurveDesignProcess}

			By observing the process that most people seem to go through when designing curves using vector graphics software, the following model appears plausible.

			\begin{enumerate}
				\item obtain the source curve as either a real or a mental image of the curve
				\item extract a set of properties from the source curve that uniquely identify it
				\item provide these properties to the software
				\item have the software turn them into the result curve
			\end{enumerate}

			Steps 2 to 3 are then repeated on portions of the curve in order to make small adjustments until the result curve is sufficiently similar to the source curve.

			Step 1 is of course independent of any software that may be used by the designer. Step 2 and 3 are strongly influenced by the choice of design tool, in that the software dictates which properties it accepts as descriptions for curves (the `language' the software uses). Finally, the success in step 4 depends solely on the ability of the software to turn descriptions of curves that are given to it into mathematical curves.

			The influence which the design tool has on the success or failure of this whole process thus depends on the following criteria.

			\begin{enumerate}
				\item Can the language describe curves succinctly and uniquely?
				\item Is it easy for humans to `read' descriptions in the language from curves?
				\item Is it easy for humans to `speak' descriptions in the language to the software?
				\item Is it possible for software to build curves from descriptions in the language?
			\end{enumerate}

			At this point it seems obvious that the choice of language is the single most influential one when it comes to the usability of curve design software. Ideally, the language should be very expressive, facilitating points 1 to 3, but it should still be possible to fulfill point 4 and derive mathematical curves from a description, a process which becomes increasingly difficult with more expressive description languages. Thus, usability analysis in the subsequent sections will focus almost exclusively on this concept of language and the key points made in the list above.

		\subsection{Existing Curve Design Tools}

			Using the model of the curve design process as well as the criteria for good curve design software that were established in the previous section, the usability of various tools available in current vector graphics software can be analyzed.

			\subsubsection{Free-Hand Curve Design}

				While many people would not consider free-hand curve design a form of vector graphics curve design, it is well-suited to demonstrate how the model of the curve design process and the concept of a communication language that the designer uses to convey the curve to the software can be used to analyze the usability of curve design tools.

				Free-hand curves are created by the designer using some kind of pointing device (for instance, a mouse or a graphics tablet), to convey a large number of points that are on the curve to the software. For the simplicity of the argument, we will assume that the graphics software will simply connect these points using straight lines. Some vector graphics software will use Bézier curves instead of straight lines, but due to the large number of points involved as well as due to the low precision of free-hand input using a pointing device, this does not make much of a difference.

				Applying the criteria from section \ref{section:CurveDesignProcess} to free-hand curve design, it is immediately obvious that hundreds of points along the curve are not a succinct way of describing a curve, so criterion 1 is definitely not met very well. Criterion 2 on the other hand seems fine, since all the points that lie on the curve are immediately apparent when looking at it. Point 3 is looking promising aswell, since it's easy enough to input points into some vector graphics software, although this is severely hindered by the fact that there are so many points that need to be specified. Criterion 4 is very well met again, since the software has no problem constructing a polygon from a few hundred points.

				Note that this only applies if the software actually uses the points in the described fashion, that is, by building a curve that goes through all the points. If, on the other hand, the software is tasked with striking a balance between a curve that goes through as many points as possible while also remaining as smooth as possible, this has to be regarded as an entirely different communication language. In this case, the points the user creates using his pointing device are merely an intermediate representation, what is actually communicated to the software is the smooth motion the user makes while using the pointing device. The success of this approach depends largely on the specific algorithm the software uses to fit a smooth curve through the given points as well as the curve primitive that is eventually used to represent the final curve.

			\subsubsection{Common Design Tools for Bézier Splines}

				Bézier splines are probably the most widely used type of curve in two-dimensional vector graphics. Bézier splines are defined piecewise using Bézier curves, which are planar parametric curves (functions \(\left[0,1\right] \rightarrow \mathbb{R}^2\)) realized using polynomials (usually cubic) in Bernstein form. A Bézier curve \(B\) of degree \(n\) consists of \(n + 1\) coefficients \(P_i \in \mathbb{R}^2\). The curve will always go through the first and the last coefficient point (\(B\left(0\right) = P_0, B\left(1\right) = P_n\)) and stay within the convex hull of the coefficient points at all other times. Bézier curves in a Bézier spline are usually pieced together in such a way as to make the tangent direction along the curve a continuous function.

				In existing vector graphics software, Bézier splines are edited by more or less directly modifying the coefficient points of the underlying Bézier curves. Most vector graphics tools have functions that allow preserving the continuity of the tangent direction by automatically modifying adjacent coefficient points accordingly. Some tools can also modify existing coefficient points in such a way that the curve passes through a point that is chosen by the user without adding more Bézier curves to the spline.

				% TODO - illustrations
				% curves that are best described in terms of curvature are difficult to put down
				%   an excessive number of control points are needed to pin down curvature
				%     compromises smoothness
				%     makes the curve hard to change
				%     somewhat okay when tracing an existing image, since the shapes are already final and the source is likely smooth
				%   examples
				%     circle arcs (may have too much curvature in the middle, no constant curvature)
				%     spirals (fluttershy's mane)
				%     lugia's back line
				% changind direction at some node usually adversely affects neighboring spline segments
				%   changing the direction on the left side of a node with a long segment to the right screws up the right side
				%     bézier splines do not have much degrees of freedom left after direction continuity is ensured, even though curvature continuity may still be violated
				%     requires additional segments, tradeoff between smoothness and locality, not much tool support for retaining as much smoothness as possible while adding more freedom
				% curvature continuity is not guaranteed
				%   curves with lots of changes in curvature sign are affected the most
				%   examples
				%     lugia's belly

			\subsubsection{Common Design Tools for Spiro Splines}

				% TODO - illustrations
				% would in principle be well-suited for circle arcs and spirals and other shapes with linearly changing curvature, but curvature can only be controlled indirectly using points
				% tangent direction and curvature can only be controlled indirectly
				%   sometimes, many nodes need to be placed to convey the desired direction/curvature information
				%   specifying direction/curvature always comes with the unwanted side-effect of specifying additional points, possibly introducing unwanted bumpiness
				% sometimes, nice, smooth curves can be achieved using only few points
				%   it's not obvious where the points have to be specified such that one needs to specify as few as possible
				%   changing those 'minimal' curves is somewhat hampered by the fact that the purpose of each point may be non-obvious
				%     has bad locality since the points on the curve don't really specify points, but have an ulterior purpose
				% suffers from instability issues
				% does guarantee curvature continuity

		\subsection{Specification-Based Curves}

			% with language being this important, it seems irresponsible to have it determined by the mathematical parameters of the model as the language
			% moving from implementation detail to differential geometric specification
			% implementation section of https://code.google.com/p/kurve/wiki/Foundation
            % TODO - diagram: process of going from user input and fairness to curve
			% TODO - illustration: visually introduce point, direction, curvature with tangents and osculating circles

	\section{Proposed Solution: Nonlinear Optimization on Polynomial Curves}

			% segmentation, separation of segmentation and specification, global positions on curve (https://code.google.com/p/kurve/wiki/SegmentationParametrization, maybe diagram)
			% disambiguation
			% specifications
			%   position, direction and curvature
			% continuity connections
			% constant velocity
			% fairness
			%   derivative of projected acceleration is equal to projected jerk

	\section{Implementation}

	\section{Evaluation}

			% demonstrate euler spirals and bézier curves being embedded
			% revisit examples made when analyzing Bézier splines and Spiro splines

	\section{Conclusion}

		\begin{table}[htbp]
			\centering
			\begin{tabular}{c|c|r|r|}
				\multicolumn{1}{c}{} & \multicolumn{1}{c}{} & \multicolumn{2}{c}{Actual Choice} \\
				\cline{2-4}
				& Payout & \multicolumn{1}{|c|}{A and B} & \multicolumn{1}{|c|}{B only} \\
				\cline{2-4}
				\multirow{2}{*}{Predicted Choice} & A and B & 1 000 \$ & 0 \$ \\
				\cline{2-4}
				& B only & 1 001 000 \$ & 1 000 000 \$ \\
				\cline{2-4}
			\end{tabular}
			\caption{Possible payouts for Newcomb's paradox}
			\label{table:NewcombParadoxPayouts}
		\end{table}

	\begin{thebibliography}{}

		\bibitem{CranePatterson2000}
			\emph{History of the mind-body problem}\\
			Tim Crane, Sarah Patterson\\
			Routledge, 2000

	\end{thebibliography}

\end{document}
