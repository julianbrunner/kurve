\documentclass[a4paper]{article}

\usepackage[utf8]{inputenc}
\usepackage[T1]{fontenc}
\usepackage{parskip}
\usepackage{times}
\usepackage{multirow}

\clubpenalty = 10000
\widowpenalty = 10000

\title{Optimal Fitting of Planar Curves to Prescribed Constraints}
\author{Julian Asamer, Julian Brunner}
\date{\today}

\begin{document}

	\maketitle

	\begin{abstract}

		\noindent TODO

	\end{abstract}

	\section{Introduction}

		These days, vector graphics are widely used in both artistic and industrial design. Artists create illustrations, character designs and even fully animated cartoons using vector graphics software. Vector graphics are also used in industrial design, allowing for the smooth, aerodynamic shapes observed in modern cars and airplanes.

		Raster graphics images (for instance, digital photos and most digitally painted artworks) consist of a finite number of pixels (picture elements) which discretely specify the colors at each point of the image. In contrast, vector graphics designs consist of primitives like curves and surfaces, which describe the shapes present in the design using continuous mathematical functions. This has various advantages over the raster approach, since designs do not suffer from discretization artifacts and can, for instance, be arbitrarily scaled without any loss of quality.

		The scope of this project is restricted to two-dimensional vector graphics, but many ideas can be applied to three-dimensional vector graphics as well. All elements in two-dimensional vector graphics are described in terms of curves, so the design process of curves is the single most important part of two-dimensional vector graphics. Unfortunately, designing curves using the tools available through current vector graphics software can be very frustrating. The artist is often unable to make the curve look the way they want it to. To make matters worse it is often unclear why this is the case, as the software seems to put all the necessary tools into the hands of the designer, yet at times, they somehow don't allow the artist to communicate their intention to the software. Thus, the objective of this project is to identify the exact nature and cause of these inadequacies and then specify and implement curve design tools that do not exhibit them.

		% TODO: make sure introduction does not overlap with start of next section regarding both content and wording

	\section{Usability of Curve Design Tools}

		When trying to improve something, it seems advisable to take a step back and thoroughly research the causes of its shortcomings so as to find a way to improve it without making the same or similar mistakes. In the context of curve design tools, this can be achieved by analyzing and modelling the process that people use when designing curves. Once such a model has been obtained, it can be used to understand the issues with current software as well as to propose criteria for good curve design tools.

		\subsection{Curve Design Process}

			By observing the process that most people seem to go through when designing curves using vector graphics software, the following model appears plausible.

			\begin{enumerate}
				\item obtain the source curve as either a real or a mental image of the curve
				\item extract a set of properties from the source curve that uniquely identify it
				\item provide these properties to the software
				\item have the software turn them into the result curve
			\end{enumerate}

			Steps 2 to 3 are then repeated on portions of the curve in order to make small adjustments until the result curve is sufficiently similar to the source curve.

			Step 1 is of course independent of any software that may be used by the designer. Step 2 and 3 are strongly influenced by the choice of design tool, in that the software dictates which properties it accepts as descriptions for curves (the `language' the software uses). Finally, the success in step 4 depends solely on the ability of the software to turn descriptions of curves that are given to it into mathematical curves.

			The influence which the design tool has on the success or failure of this whole process thus depends on the following key points.

			\begin{itemize}
				\item Can the language describe curves succinctly and uniquely?
				\item Is it easy for humans to `read' the language from curves?
				\item Is it easy for humans to `speak' the language to the software?
				\item Is it possible for software to build curves from descriptions in the language?
			\end{itemize}

			At this point it seems obvious that the choice of language is the single most influential one when it comes to the usability of curve design software. Thus, usability analysis in the subsequent sections will focus almost exclusively on this concept of language and the key points made in the list above.

		\subsection{Shortcomings of Existing Curve Design Tools}

			Using the model of the curve design process established in the previous section, various tools available in current vector graphics software can be analyzed 

			\subsubsection{Free-Hand Curves}

			\subsubsection{Bézier Splines}

				

			\subsubsection{Spiro Splines}

			% shortcomings of existing solutions (bézier, spiro (link to spiro promotion), NURBS)
			%   TODO: maybe correspond these shortcomings and their illustrations to the stuff in the evalution section
			%   TODO: research shortcomings that spiro splines do not solve, maybe we can have illustrations for that, too
			%   illustration: it's easy to create parabolas instead of circle arcs
			%   illustration: spirals need many control points and it's easy to include small bumps
			%   illustration: curvature continuity is not guaranteed

		\subsection{Criteria for Good Cruve Design Tools}

			% https://code.google.com/p/kurve/wiki/Foundation
			% TODO - illustration: visually introduce point, direction, curvature with tangents and osculating circles

	\section{Proposed Solution}

		\subsection{Specification-Based Curves}

			% with language being this important, it seems irresponsible to simply choose to use the mathematical parameters of the model as the language
			% moving from implementation detail to differential geometric specification
			% implementation section of https://code.google.com/p/kurve/wiki/Foundation

		\subsection{Nonlinear Optimization on Polynomial Curves}

			% segmentation, separation of segmentation and specification, global positions on curve (https://code.google.com/p/kurve/wiki/SegmentationParametrization, maybe diagram)
			% disambiguation
			% specifications
			%   position, direction and curvature
			% continuity connections
			% constant velocity
			% fairness
			%   derivative of projected acceleration is equal to projected jerk

	\section{Implementation}

	\section{Evaluation}

	\section{Conclusion}

		\begin{table}[htbp]
			\centering
			\begin{tabular}{c|c|r|r|}
				\multicolumn{1}{c}{} & \multicolumn{1}{c}{} & \multicolumn{2}{c}{Actual Choice} \\
				\cline{2-4}
				& Payout & \multicolumn{1}{|c|}{A and B} & \multicolumn{1}{|c|}{B only} \\
				\cline{2-4}
				\multirow{2}{*}{Predicted Choice} & A and B & 1 000 \$ & 0 \$ \\
				\cline{2-4}
				& B only & 1 001 000 \$ & 1 000 000 \$ \\
				\cline{2-4}
			\end{tabular}
			\caption{Possible payouts for Newcomb's paradox}
			\label{table:NewcombParadoxPayouts}
		\end{table}

	\begin{thebibliography}{}

		\bibitem{CranePatterson2000}
			\emph{History of the mind-body problem}\\
			Tim Crane, Sarah Patterson\\
			Routledge, 2000

	\end{thebibliography}

\end{document}
